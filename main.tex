% Key Features:
% -------------
% - Customizable sections: Education, Research Experience, Awards, Publications, etc.
% - Bookmarks in the PDF for easy navigation
% - Styled bibliography with BibLaTeX
% - Hyperlinked email and website
% - FontAwesome icons for additional styling
%
% Getting Started:
% ----------------
% 1. Customize your personal information by modifying the \mytitle command.
% 2. Add your content to the respective subfiles (e.g., education.tex, exp_research.tex).
% 3. Update the bibliography file (ref.bib) with your publications and categorize them with keywords.

\documentclass[11pt]{article} % Choose the document class and font size
\usepackage[margin=1in]{geometry} % Page layout settings

% Set the citation style
\usepackage[
    backend=biber,      % Specifies the backend to be used by BibLaTeX for processing the bibliography. 'biber' is the default backend.
    maxnames=20,        % Limits the maximum number of author names to display before abbreviating with "et al."
    style=nature,       % Sets the citation style to 'nature,' which is commonly used in scientific papers.
    sorting=ydnt,       % Specifies the sorting order of entries in the bibliography:
                        % y - year (descending)
                        % d - descending order
                        % n - name
                        % t - title
    defernumbers=true,  % Delays the assignment of citation numbers until the end of the document, allowing for the correct order of citations within each bibliography section.
]{biblatex}
\addbibresource{ref.bib} % Adds the bibliography resource file 'ref.bib' containing all the references.

%template packages
\usepackage{longtable}
\usepackage{bookmark}
\usepackage{fontawesome}
\usepackage{ragged2e}
\usepackage{soul}
\usepackage{kpfonts} % More professional font
% \usepackage[default]{sourcecodepro} % Code-like font
\usepackage[T1]{fontenc}

% my packages
\usepackage{moderncv}

% Control hyperlinks and colors
% CUSTOM COLORS INCLUDED DIRECTLY AFTER \begin{document}
\usepackage{xcolor}
\usepackage{hyperref}
\hypersetup{
    colorlinks=true,        % Enable colored links
    breaklinks=true,        % Allow links to break across lines
    linkcolor=cornflowerblue,    % Color of internal links
    urlcolor=cornflowerblue,     % Color of URL links
    anchorcolor=cornflowerblue,  % Color of anchors
    citecolor=cornflowerblue,    % Color of citations
    pdftitle={Your Title},    % Title of the PDF
    pdfauthor={Your Name}, % Author of the PDF
    bookmarksopen=true,      % Open bookmarks panel at start
}

%%% CONVENIENCE FUNCTIONS GO HERE %%%
%%% ----------------------------- %%%
\newcommand{\mytitle}[4]{
  \begin{center}
    \Large\textbf{#1}\normalsize \\ % Name in large bold font
    \href{mailto:#2}{#2} \\ % Email with mailto: link
    \href{https://#3}{#3} \\ % Website with link
    #4 % Address
  \end{center}
}

\newcommand{\myinput}[1]{
  \input{include/#1}
}
%%% ----------------------------- %%%


\begin{document}
% Set custom colors here (imported directly after \begin{document})
% The below use HTML hex codes.
% More HTML hex codes: https://encycolorpedia.com/html
\definecolor{firebrick}{HTML}{b22222} 
\definecolor{darkslategrey}{HTML}{2f4f4f} 
\definecolor{cornflowerblue}{HTML}{6495ed} 
\definecolor{mediumslateblue}{HTML}{7b68ee}  % Load custom colors from colors file
\mytitle{
  Qile Yang \textcolor{gray}{| Curriculum Vitae}
}{
  qile.yang@berkeley.edu
}{
  https://qile0317.github.io/
}{
  Placeholder
  %\social[linkedin]{qile0317}
}


% Ensure right side margin is not surpassed by bibliography and the right margin is aligned throughout
\RaggedRight


% These \pdfbookmark lines create bookmarks in the exported PDF document that display in the left pane.
% Value in [] sets the indentation level of the bookmark
\pdfbookmark[1]{Education}{}
\section*{Education}
\myinput{education.tex}

\pdfbookmark[1]{Research Experience}{exp_research}
\section*{Research Experience}
\label{exp_research}
\myinput{exp_research.tex}

\pdfbookmark[1]{Awards \& Honors}{awards}
\section*{Awards \& Honors}
\label{awards}
\myinput{awards.tex}


\pdfbookmark[1]{Publications}{pubs}
\section*{Publications}
\label{pubs}

% Add equal contribution dagger
\vspace{-.75em}
\small
\faGoogle~\href{https://scholar.google.com/}{Google Scholar}\\
$\dagger \rightarrow$ Equal contribution
\normalsize


\pdfbookmark[2]{Journal Articles}{journal-article}
\subsection*{Journal Articles}
\label{journal-article}
\newrefcontext[labelprefix=J] % Will prefix bibliography numbers with this letter
% Ensures publications which are not cited in the document are included in the above sections
\nocite{*} % Ensures uncited items are included
\printbibliography[
    type=article, % Only include @article ref.bib items
    heading=none, % Do not include header. Gives us more control.
    resetnumbers=true, % Start item counter from zero
    keyword=J % Include items in ref.bib with keyword={J}
]

\pdfbookmark[2]{Peer-reviewed Conference Proceedings}{conferences}
\subsection*{Peer-reviewed Conference Proceedings}
\label{conferences}
\newrefcontext[labelprefix=C]
\printbibliography[type=inproceedings,heading=none,resetnumbers=true,keyword=C]

\pdfbookmark[2]{Working papers}{working-papers}
\subsection*{Working papers}
\label{working-papers}
\newrefcontext[labelprefix=W]
\printbibliography[type=misc,heading=none,resetnumbers=true,keyword=R]
    

\pdfbookmark[1]{Tools \& Software}{tools}
\section*{Tools \& Software}
\label{tools}
\myinput{tools.tex}


\pdfbookmark[1]{Presentations}{presentations}
\section*{Presentations}
\label{presentations}

% Include any additional details here
% \vspace{-.75em}
% \small
% $\dagger \rightarrow$ Equal contribution
% \normalsize

\pdfbookmark[2]{Talks}{talks}
\subsection*{Talks}
\label{talks}
\newrefcontext[labelprefix=T]
\printbibliography[type=misc,heading=none,resetnumbers=true,keyword=T]

\pdfbookmark[2]{Posters}{posters}
\subsection*{Posters}
\label{posters}
\newrefcontext[labelprefix=P]        
\printbibliography[type=misc,heading=none,resetnumbers=true,keyword=P]

\pdfbookmark[2]{Demonstrations \& Tutorials}{demos}
\subsection*{Demonstrations \& Tutorials}
\label{demos}
\newrefcontext[labelprefix=D] \printbibliography[type=misc,heading=none,resetnumbers=true,keyword=D]


\pdfbookmark[1]{Selected Media Coverage}{media}
\section*{Selected Media Coverage}
\label{media}
\myinput{media.tex} % Input lines load the material from the subdocuments


\pdfbookmark[1]{Teaching}{teaching}
\section*{Teaching}
\label{teaching}
\myinput{teaching.tex}


\pdfbookmark[1]{Academic Advising}{advising}
\section*{Academic Advising}
\label{advising}
\myinput{advising.tex}


\pdfbookmark[1]{Academic Service}{service}
\section*{Academic Service}
\label{service}
\myinput{service.tex}


\pdfbookmark[1]{Other Experience}{exp_other}
\section*{Other Experience}
\label{exp_other}
\myinput{exp_other.tex}


% Pretty ending with the date last updated
\centering
\rule{0.25\linewidth}{0.4pt}\\
\medskip
Last updated: \today

\end{document}
